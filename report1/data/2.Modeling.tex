\newpage

\section{任务介绍}

\subsection{建模对象描述}

所选定建模对象的系统特性将从集合性、相关性、环境适应性三个方面来分析。

\subsubsection{集合性}

该系统由执行机构(机械子系统)和驱动机构(液压子系统)组成,二者缺一不可。
 
所谓执行机构在这里指的是机械系统,主要是由铰接点连接的连杆结构。我们可以参照动物相关生理结构对结构进行命名,参见图 1-3。这里值得一提的是踝关节设计为被动式,踝关节和小腿组成一组移动副,在小腿中有一个沿小腿轴线布置的弹簧,弹簧有助于减少足部接触地面时对机器人的冲击力。同时,在足部外面包裹了一层弹性橡胶。这一层橡胶有助于减小足部在地面上的摩擦,从而改善腿部牵引能力。总的来说,该系统能够实现髋关节和膝关节的弯曲和伸展,以及踝关节的直线滑动。 

所谓驱动机构是指为机械系统提供动力的部分,这里主要指液压系统(图 1-4)。液压系统由液压泵、阀、液压缸等组成,其中液压缸是直接与机械系统相连、为机械系统提供动力的元件。由液压泵将机械能转化为液压能。通过三位四通换向阀改变液压缸中液压油的输入和输出,从而改变液压缸活塞的移动方向,控制髋关节和膝关节是弯曲还是伸展,或者保持现有状态。溢流阀起安全保护作用。蓄能器可以作辅助能源,必要时提供额外的流量,并且可以减少压力脉动、吸收液压冲击。换交换器使液压油保持在适宜温度。

\subsubsection{相关性}

系统内每一子系统相互依存、相互制约、相互作用而形成了一个相互关联的整体。

机械子系统和液压子系统之间的联系是通过液压缸活塞杆和连杆连接点具有共同的位置和速度来实现的。通过液压缸的运动,实现连杆相对于机架的转动。子系统连接分析见图 1-5。根据几何计算,我们可以得到转动角度 q1、q2 和各部分长度的关系。这里我们液压子系统和机械子系统之间相互关联的变量是铰接点转动的角速度,也就是说液压子系统给机械子系统的输入变量是铰接点转动的角速度,这样便于和模型仿真相一致。 

\subsubsection{环境适应性}

和其他移动机器人相比,该液压四足机器人具有较强的环境适应性。在移动机器人领域中,经典的轮式或履带式系统能够非常有效地处理平坦且结构良好的固体表面(例如实验室和道路)然而,轮式机器人在崎岖不平的地形上仍然存在着很大的局限性。研究人员在过去几十年里专注于仿生腿式机器的建造。绝大多数现有的腿式机器人是由电动驱动,因为它的大小、价格合适,易于使用和控制。然而,电动机的功率-质量比以及力-质量比小,需要牺牲速度能力来获取大扭矩输出。而且每次机器人足部撞击地面时会产生瞬时高扭矩峰值,容易造成减速箱齿轮断裂。 

该液压四足机器人利用液压来改善机器人的动力特性。液压传动具有功率-质量比以及力-质量比大、负载刚度大、调速范围大等优点,能适应机器人在崎岖路面和复杂工况下工作的需要。小腿中的弹簧和足部的橡胶可以减小机器人足部撞击地面时带来的瞬时冲击力。

\subsection{建模问题提取}

可以将对象的工况分为两种情况:
1.在平坦且结构良好的 平坦且结构良好的 平面道路条件下:
在该条件下,机器人底部受到的速度输入较小,躯干可以维持较为稳定的姿态,但需要配置好系统的工作参数。 
2.在崎岖不平且不坚固的道路条件下:
在该条件下,机器人底部受到的速度输入较大,躯干如何响应外界速度输入,系统如何做出响应,是我们主要探讨的问题。 

\subsection{模型假设与标记}

\subsubsection{模型假设}

针对上述工况,结合简化模型的角度出发,做出如下假设:

\begin{itemize}
	
	\item 假设机械各部分零件为刚体;
	
	\item 假设刚体质心位于几何中心;
	
	\item 忽略铰接点连接处的摩擦;
	
	\item 忽略电驱动系统工作产热带来的参数变化。
	
\end{itemize}

\subsubsection{模型标记}

表 \ref{tab:param} 列出本模型的相关参数及其数学标记。主要分为分为整体系统参数,手推轮椅主体,机电驱动模块以及手推控制输入部分的参数,如下所示:

\begin{table}[htpb]
\caption{系统主要参数及其数学标记}\label{tab:param}
\begin{longtable}{l|l}
	\toprule
	\textbf{数学标记} & \textbf{系统参数}\\
	\midrule
	\endhead
	\multicolumn{2}{l}{\textbf{系统整体参数}} \\ % {占用行数} {文字居左中右} {内容}
	\midrule
	$ M $ & 总体质量\\
	$ J $ & 总体转动惯量\\
	$ V_{\rm{CG}} $ & 质心速度\\
	$ w_{\rm{CG}} $ & 质心转速\\
	\midrule
	\multicolumn{2}{l}{\textbf{手推轮椅主体}} \\
	\midrule
	$ \tau_{\rm{MSE}} $ & 后轮推进扭矩 \\
	$J w$ & 后轮转动惯量\\
	$ M_t $ & 系统质量\\
	$J t$ & 系统转动惯量\\
	$R_g$ & 轮胎与地面摩擦系数\\
	$ C_w $ & 轮辐弹性系数\\
	$R_w$ & 轮辐阻尼\\
	$ r $ & 后轮半径\\
	$ L_w $ & 轮椅宽度\\
	\midrule
	\multicolumn{2}{l}{\textbf{机电驱动模块}} \\
	\midrule
	$ I_e $ & 电机电感\\
	$ R_e $ & 电机内阻\\
	$ J_r $ & 转子转动惯量\\
	$ R_b $ & 电机轴承阻尼\\
	$ M_d $ & 机电驱动模块质量\\
	$ J_d $ & 机电驱动模块转动惯量\\
	$ L_d $ & 模块宽度\\
	$C_s $ & 电机输出转矩\\
	$ R_s $ & 电机输出轴阻尼\\
	$ k_1,\ k_8 $ & 电机扭矩系数Motor torque constant\\
	$ k_2,\ k_7 $ & 齿轮师叔比\\
	$ k_3,\ k_6 $ & Motorized wheels radius\\
	$ L_{\rm{MSe}} $ & 输入控制电压\\
	\midrule
	\multicolumn{2}{l}{\textbf{手推控制输入部分}}\\
	\midrule
	$ J $ & 手动轮转动惯量\\
	$ k_9,\ k_{10} $ & 手动论半径\\
	\bottomrule
\end{longtable}
\end{table}