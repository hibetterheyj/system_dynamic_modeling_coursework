\clearpage
\section{实验阶段}

复杂和混合域动态系统的模型可以在最常用的计算机软件上进行模拟,这些软件提供内置的数学和图形功能,如Matlab / Simulink,Modelec,Acsyde和Matrix / SystemBuild。 本节描述了将键合图模型转换为框图时所遵循的步骤,并使用Matlab / Simulink来模拟键合图模型。

\subsection{实验参数设定}

表 \ref{tab:param} 列出本模型的相关参数及其数学标记。本报告中表示左右参数则分别以 $ \_L $ 和 $ \_R $表示。参数表主要分为分为整体系统参数,手推轮椅主体,机电驱动模块以及手推控制输入部分的参数四个部分。如下所示:

\begin{table}[H]
	\footnotesize
	\caption{系统主要参数及其数学标记}\label{tab:param}
	\begin{longtable}{l|l|l}
		\toprule
		\textbf{数学标记} & \textbf{系统参数} & \textbf{设计值}\\
		\midrule
		\endhead
		\multicolumn{3}{l}{\textbf{系统整体参数}} \\ % {占用行数} {文字居左中右} {内容}
		\midrule
		$ V_{\rm{CG}} $ & 质心速度 & \\
		$ w_{\rm{CG}} $ & 质心转速 & \\
		$ P_{\rm{CG}} $ & 系统动量变化率 & \\
		$ P_{\theta} $ & 系统角动量变化率 & \\
		\midrule
		\multicolumn{3}{l}{\textbf{手推轮椅主体}} \\
		\midrule
		SE:$\tau$ & 后轮推进扭矩($ \mathrm{N} \cdot \mathrm{m} $) & 12 \\
		$ J w $ & 后轮转动惯量($ \mathrm{kg} \cdot \mathrm{m}^2 $) & 0.005 \\
		$ M_t $ & 系统质量 & 100 \\
		$ J t $ & 系统转动惯量 & 64 \\
		$ R_g $ & 轮胎与地面摩擦系数 & 0.006 \\
		$ C_w $ & 轮辐弹性系数 & 0.0021 \\
		$ R_w $ & 轮辐阻尼 & 12 \\
		$ r $ & 后轮半径 & 0.3 \\
		$ L_w $ & 轮椅宽度 & 0.8 \\
		$ R_{wc} $ & 联轴器阻尼系数 & 0.005 \\
		$ C_{wc} $ & 联轴器弹性系数 & 0.005 \\
		\midrule
		\multicolumn{3}{l}{\textbf{机电驱动模块}} \\
		\midrule
		$ I_e $ & 电机电感 & 0.0033 \\
		$ R_e $ & 电机内阻 & 0.9 \\
		$ J_r $ & 转子转动惯量 & 0.078 \\
		$ R_b $ & 电机轴承阻尼 & 0.008 \\
		$ M_d $ & 机电驱动模块质量 & 30 \\
		$ J_d $ & 机电驱动模块转动惯量 & 75 \\
		$ L_d $ & 模块宽度 & 0.6 \\
		$ C_s $ & 电机输出转矩 & 0.00237\\
		$ R_s $ & 电机输出轴阻尼 & 11 \\
		$ k_1 $ & 电机扭矩系数 & 0.288 \\ % 简化了k8
		$ k_2 $ & 齿轮系数比 & 0.18 \\ % 简化了k7
		$ k_3 $ & 电动车轮半径 & 0.127 \\ % 简化了k6
		Se:L & 输入控制电压 & 24 \\
		\midrule
		\multicolumn{3}{l}{\textbf{手推控制输入部分}}\\
		\midrule
		$ J $ & 手动轮转动惯量 & 0.005 \\
		$ k_4 $ & 手动轮半径 & 0.3 \\ % 简化了k10
		\bottomrule
	\end{longtable}
\end{table}

\clearpage

\subsection{Simulink相关模块运用和操作}

在建模过程中,我们运用了simulink一些内置的模块和操作进行建模,以下是简要的介绍:

\begin{itemize}
	
	\item 生成常量值 Constant
	
	%%%%%%%%%%%%%%%%%
	\begin{figure}[H]
		\centering
		\includegraphics[width=0.2\textwidth]{fig/simulink/constant_block.png}
		\caption{常数模块图例}\label{fig:constant_block}
	\end{figure}
	%%%%%%%%%%%%%%%%%
	
	Constant 模块生成实数或复数常量值。在本次建模中,主要用于常值输入以及一些固定参数的表示。
	
	\item 将输入乘以常量 Gain
	
	%%%%%%%%%%%%%%%%%
	\begin{figure}[H]
		\centering
		\includegraphics[width=0.2\textwidth]{fig/simulink/gain_block.png}
		\caption{增益模块图例}\label{fig:gain_block}
	\end{figure}
	%%%%%%%%%%%%%%%%%
	
	Gain 模块将输入乘以一个常量值(增益)。输入和增益可以是标量、向量或矩阵。在本次建模中,主要用于表示状态空间方程中,状态变量前面的系数。
	
	\item 输入信号的加减运算 Sum
	
	%%%%%%%%%%%%%%%%%
	\begin{figure}[H]
		\centering
		\includegraphics[width=0.2\textwidth]{fig/simulink/sum_block.png}
		\caption{加减运算模块图例}\label{fig:sum_block}
	\end{figure}
	%%%%%%%%%%%%%%%%%
	
	Sum 模块对输入信号执行加减运算。Add、Subtract、Sum of Elements 和 Sum 模块是相同的模块。此模块可对标量、向量或矩阵输入执行加减运算。它还可以缩减信号的元素并执行求和。在本次建模中,主要用于表示状态空间方程中的加减运算。
	
	\item 可自定义函数模块 MATLAB Function
	
	%%%%%%%%%%%%%%%%%
	\begin{figure}[H]
		\centering
		\includegraphics[width=0.2\textwidth]{fig/simulink/matlab_function_block.png}
		\caption{可自定义函数模块图例}\label{fig:matlab_function_block}
	\end{figure}
	%%%%%%%%%%%%%%%%%
	
	使用 MATLAB Function 模块可以编写用于 Simulink模型的 MATLAB函数。在本次建模中,主要用于表示较为复杂的判断与运算功能,给仿真带来一定便利。
	
	\item 饱和模块 Saturation
	
	将输入信号限制在饱和上界和下界值之间
	
	%%%%%%%%%%%%%%%%%
	\begin{figure}[H]
		\centering
		\includegraphics[width=0.2\textwidth]{fig/simulink/saturation_block.png}
		\caption{饱和模块图例}\label{fig:saturation_block}
	\end{figure}
	%%%%%%%%%%%%%%%%%
	
	Saturation 模块产生输出信号,该信号是在饱和上界和下界值之间的输入信号值。上界和下界由参数 Upper limit 和 Lower limit 指定。在本次建模中,主要用于限制相关变量的最大值,避免Inf无穷大值的发生而无法仿真。
	
	\item 显示输入常数值模块 Display
	
	%%%%%%%%%%%%%%%%%
	\begin{figure}[H]
		\centering
		\includegraphics[width=0.2\textwidth]{fig/simulink/display_block.png}
		\caption{显示输入常数值模块图例}\label{fig:display_block}
	\end{figure}
	%%%%%%%%%%%%%%%%%
	
	Display 模块显示输入数据的值。您可以指定显示的格式和频率。在本次建模中,主要用于判断输入值是否合理,从而保证仿真的正常进行。
	
	\item 示波器 Scope
	
	%%%%%%%%%%%%%%%%%
	\begin{figure}[H]
		\centering
		\includegraphics[width=0.1\textwidth]{fig/simulink/scope_block.png}
		\caption{示波器图例}\label{fig:scope_block}
	\end{figure}
	%%%%%%%%%%%%%%%%%
	
	Simulink Scope 模块显示时域信号。在本次建模中,主要用于输出随时间变化的状态变量,如速度,角速度等等。
	
	\item 合并变量模块 Mux
	
	将相同数据类型和数值类型的输入信号合并为虚拟向量
	
	%%%%%%%%%%%%%%%%%
	\begin{figure}[H]
		\centering
		\includegraphics[width=0.1\textwidth]{fig/simulink/mux_block.png}
		\caption{合并变量模块图例}\label{fig:mux_block}
	\end{figure}
	%%%%%%%%%%%%%%%%%
	
	Mux 模块可将其输入合并为单个向量输出。输入可以是标量或向量信号。所有输入都必须具有相同的数据类型和数值类型。在本次建模中,主要用于合并输出结果,后续输出至工作空间。
	
	\item 写入工作区模块 To Workspace
		
	将相同数据类型和数值类型的输入信号合并为虚拟向量
	
	%%%%%%%%%%%%%%%%%
	\begin{figure}[H]
		\centering
		\includegraphics[width=0.2\textwidth]{fig/simulink/to_workspace_block.png}
		\caption{写入工作区模块图例}\label{fig:to_workspace_block}
	\end{figure}
	%%%%%%%%%%%%%%%%%
	
	To Workspace 模块将输入信号数据写入到工作区。在仿真期间,模块将数据写入到内部缓冲区。暂停仿真或仿真完成后,该数据将写入到工作区。在仿真暂停或停止之前,数据不可用。在本次建模中,主要将输出后结果,后续输出至工作空间进行处理作图。
	
	\item 创建模块封装 Subsystem mask
	
	为子系统和自定义模块创建自定义外观、创建用户定义的界面、封装逻辑以及隐藏数据。封装是用于模块的一种自定义用户界面。
	
	在对部分元件进行子系统封装后,可以进行Mask编辑:
	
	%%%%%%%%%%%%%%%%%
	\begin{figure}[H]
		\centering
		\includegraphics[width=0.85\textwidth]{fig/simulink/mask_operation.png}
		\caption{子系统封装操作图例}\label{fig:mask_operation}
	\end{figure}
	%%%%%%%%%%%%%%%%%
	
	在具体Mask界面中,可以直接添加相关参数编辑模块:
	
	%%%%%%%%%%%%%%%%%
	\begin{figure}[H]
		\centering
		\includegraphics[width=0.85\textwidth]{fig/simulink/mask_interface.png}
		\caption{相关参数编辑操作图例}\label{fig:mask_interface}
	\end{figure}
	%%%%%%%%%%%%%%%%%
	
	在本次建模中,主要用于封装模型,以及快速参数调整。最终实现效果如下:
	
	%%%%%%%%%%%%%%%%%
	\begin{figure}[H]
		\centering
		\includegraphics[width=0.85\textwidth]{fig/simulink/block_params.png}
		\caption{参数调整界面图例}\label{fig:block_params}
	\end{figure}
	%%%%%%%%%%%%%%%%%
	
	\clearpage
	
	\subsection{手推轮椅主体仿真}
	
	\subsubsection{MPW 整体框图介绍}
	
	%%%%%%%%%%%%%%%%%
	\begin{figure}[htbp]
		\centering
		\includegraphics[width=0.9\textwidth]{fig/simulink/MPW_simulink1.png}
		\caption{MPW 整体框图}\label{fig:MPW_simulink}
	\end{figure}
	%%%%%%%%%%%%%%%%%
	
	其中输入状态变量子系统主要由八个状态空间方程构成部分组成,	输出状态变量子系统由四个增益组成,如下所示:
	%%%%%%%%%%%%%%%%%
	\begin{figure}[htbp]
		\centering
		\includegraphics[width=0.88\textwidth]{fig/simulink/MPW_simulink_block1.png}
		\caption{MPW 输入状态参量子系统-block1}
	\end{figure}
	%%%%%%%%%%%%%%%%%
	%%%%%%%%%%%%%%%%%
	\begin{figure}[ht]
		\centering
		\includegraphics[width=0.9\textwidth]{fig/simulink/MPW_simulink_block2.png}
		\caption{MPW 输入状态参量子系统-block2}
	\end{figure}
	%%%%%%%%%%%%%%%%%
	%%%%%%%%%%%%%%%%%
	\begin{figure}[ht]
		\centering
		\includegraphics[width=0.9\textwidth]{fig/simulink/MPW_simulink_block3.png}
		\caption{MPW 输入状态参量子系统-block3}
	\end{figure}
	%%%%%%%%%%%%%%%%%
	%%%%%%%%%%%%%%%%%
	\begin{figure}[b]
		\centering
		\includegraphics[width=0.75\textwidth]{fig/simulink/MPW_simulink_block4.png}
		\caption{MPW 输出状态变量子系统}
	\end{figure}
	%%%%%%%%%%%%%%%%%
	
	\clearpage
	
	此外, rotation\_radius\_compution 函数代码如下所示:
	%%%%%%%%%%%%%%%%%
	\begin{figure}[htbp]
		\centering
		\includegraphics[width=0.9\textwidth]{fig/simulink/code.png}
		\caption{MPW 中 rotation\_radius\_compution 函数代码}\label{fig:code}
	\end{figure}
	%%%%%%%%%%%%%%%%%
	
	%\clearpage
	\subsubsection{MPW 仿真与分析}
	
	表 \ref{tab:param} 给出了元件的一般特征和数值。在推进过程中施加在手边缘上的力是正弦的,包括在推动开始时的快速加载速率,导致冲击尖峰,然后逐渐施加和释放力
	为了本研究的目的,系统用步进输入信号(代表手动扭矩)进行模拟,以产生避障预定轨迹。
	主要目的是说明没有MDM参与的MPW系统的行为。
	
	如图所示,在两个轮上施加 2$ N \cdot m $ 范围内的阶跃信号的组合。
	%%%%%%%%%%%%%%%%%
	\begin{figure}[!h]
		\centering
		\includegraphics[width=0.58\textwidth]{fig/simulation/MPW_tauL.pdf}
		\caption{MPW 输入扭矩}\label{MPW_tauL}
	\end{figure}
	%%%%%%%%%%%%%%%%%
	
	观察位置和取向(参考x轴)系统在直线方向上产生一些运动,并导致后轮速度,质心处速度逐渐增大,如图所示。不过观察到,应该为0的质心处角速度慢慢上升,可能为系统仿真误差导致。
	%%%%%%%%%%%%%%%%%
	\begin{figure}[!h]
		\centering
		\includegraphics[width=0.56\textwidth]{fig/simulation/MPW_wL.pdf}
		\caption{MPW 后轮输出角速度}\label{MPW_wL}
	\end{figure}
	%%%%%%%%%%%%%%%%%
	%%%%%%%%%%%%%%%%%
	\begin{figure}[!h]
		\centering
		\includegraphics[width=0.56\textwidth]{fig/simulation/MPW_vCG.pdf}
		\caption{MPW 质心处速度}\label{MPW_vCG}
	\end{figure}
	%%%%%%%%%%%%%%%%%
	%%%%%%%%%%%%%%%%%
	\begin{figure}[!h]
		\centering
		\includegraphics[width=0.56\textwidth]{fig/simulation/MPW_wCG.pdf}
		\caption{MPW 质心处角速度}\label{MPW_wCG}
	\end{figure}
	%%%%%%%%%%%%%%%%%
	
	\clearpage
	\subsection{电驱动模块仿真}
	
	\subsubsection{MDM 整体框图介绍}
	%%%%%%%%%%%%%%%%%
	\begin{figure}[htbp]
		\centering
		\includegraphics[width=0.9\textwidth]{fig/simulink/MDM_simulink1.png}
		\caption{MDM 整体框图}\label{fig:MDM_simulink}
	\end{figure}
	%%%%%%%%%%%%%%%%%
	
	其中输入状态变量子系统主要由十二个状态空间方程构成部分组成,输出状态变量子系统由四个增益组成,如下所示:
	%%%%%%%%%%%%%%%%%
	\begin{figure}[!h]
		\centering
		\includegraphics[width=0.75\textwidth]{fig/simulink/MDM_simulink_block1.png}
		\caption{MDM 输入状态参量子系统-block1}
	\end{figure}
	%%%%%%%%%%%%%%%%%
	%%%%%%%%%%%%%%%%%
	\begin{figure}[!h]
		\centering
		\includegraphics[width=0.75\textwidth]{fig/simulink/MDM_simulink_block2.png}
		\caption{MDM 输入状态参量子系统-block2}
	\end{figure}
	%%%%%%%%%%%%%%%%%
	%%%%%%%%%%%%%%%%%
	\begin{figure}[!h]
		\centering
		\includegraphics[width=0.75\textwidth]{fig/simulink/MDM_simulink_block3.png}
		\caption{MDM 输入状态参量子系统-block3}
	\end{figure}
	%%%%%%%%%%%%%%%%%
	%%%%%%%%%%%%%%%%%
	\begin{figure}[!h]
		\centering
		\includegraphics[width=0.75\textwidth]{fig/simulink/MDM_simulink_block4.png}
		\caption{MDM 输入状态参量子系统-block4}
	\end{figure}
	%%%%%%%%%%%%%%%%%
	%%%%%%%%%%%%%%%%%
	\begin{figure}[!t]
		\centering
		\includegraphics[width=0.65\textwidth]{fig/simulink/MDM_simulink_block5.png}
		\caption{MDM 输出状态变量子系统}
	\end{figure}
	%%%%%%%%%%%%%%%%%
	
	\clearpage
	\subsubsection{MDM 仿真与分析}
	在整个仿真过程中,考虑左右轮情况相同,则给出左轮相关参数变化图。
	
	如下图显示了输入电压$ v_L $
	%%%%%%%%%%%%%%%%%
	\begin{figure}[!h]
		\centering
		\includegraphics[width=0.52\textwidth]{fig/simulation/MDM_v_L.pdf}
		\caption{MDM 输入电压}\label{MDM_v_L}
	\end{figure}
	%%%%%%%%%%%%%%%%%
	
	MDM系统通过调节左右轮电动机的输入恒定电压,进而调节电动机的输出扭矩,由仿真图片可以知道,一开始的时候由于电动机的启动过程电流较大,使得做左右轮的扭矩出现一个突变,待电流稳定后,电动机的扭矩恒定,并作用在左右轮上使其产生角速度,MDM系统整体质心的线速度和角速度受左右轮运动的影响。由于一直有恒定电压输入,稳定时电动机有恒扭矩输出,如图 \ref{MDM_w_L} 和 \ref{MDM_vCG} 所示。。
	所以系统质心速度和角速度恒加速度增加,如图 \ref{MDM_vCG} 和 \ref{MDM_wCG} 所示。
	%%%%%%%%%%%%%%%%%
	\begin{figure}[H]
		\centering
		\includegraphics[width=0.62\textwidth]{fig/simulation/MDM_T_L.pdf}
		\caption{MDM 输出扭矩}\label{MDM_T_L}
	\end{figure}
	%%%%%%%%%%%%%%%%%
	
	%%%%%%%%%%%%%%%%%
	\begin{figure}[H]
		\centering
		\includegraphics[width=0.62\textwidth]{fig/simulation/MDM_w_L.pdf}
		\caption{MDM 后轮输出转速}\label{MDM_w_L}
	\end{figure}
	%%%%%%%%%%%%%%%%%
	
	%%%%%%%%%%%%%%%%%
	\begin{figure}[H]
		\centering
		\includegraphics[width=0.62\textwidth]{fig/simulation/MDM_vCG.pdf}
		\caption{MDM 质心处速度}\label{MDM_vCG}
	\end{figure}
	%%%%%%%%%%%%%%%%%
	
	%%%%%%%%%%%%%%%%%
	\begin{figure}[H]
		\centering
		\includegraphics[width=0.62\textwidth]{fig/simulation/MDM_wCG.pdf}
		\caption{MDM 质心处角速度}\label{MDM_wCG}
	\end{figure}
	%%%%%%%%%%%%%%%%%
	
	\clearpage
	\subsection{仿真条件详细说明} 
	
	我们采用 ode15s 解法器而不采用刚性系统常用的 ode113 解法器。 
	
	Ode113是一种变阶次多步 Adams-Bashforth-Moutlon 算法,此基于用前几个节点的值来计当 算法,此基于用前几个节点的值来计当 算法,此基于用前几个节点的值来计当 前节点的解,因此在相同精读下比 ode45和 ode23更快,比较适用于高阶或者需要大量计算的问题更快,比较适用于高阶或者需要大量计算的问题不适合于连续的系统。
	
	MATLAB推荐,ode45是大多数情况下的首选解法器。ode45 是基于 Dormand-Prince 4-5runge-kutta 公式,适用于一般非刚性系统的首选解法器。其精度较高,相比于其他解法器为中等水平。我们系统在 ode45解法器的仿真下,速度较慢且有时会出现无穷大溢出现象,故ode45解法器也无法满足我们的系统。
	
	而 ode15s 是基于数值微分公式 (NDF) 的变阶求解器。NDF 与后向差分公式(BDF,也称为 Gear 方法)有关,但比后者更高效。ode15s 求解器以数值方式生成 Jacobian 矩阵。
	在matlab建议下,如果怀疑某个问题是刚性问题,或者 ode45 失败或效率极其低下,请尝试 ode15s。经过测试,我们系统在 ode15s 解法器的仿真下,可以正常运行得到结果并且速度不慢故 ode15s 解法器可以满足我们的系统。
	
	\subsection{本章小结}
	
	在本章中,我们介绍了MPW和MDM的simulink模型的建模,分析和模拟工作及其与普通手动推进轮椅的耦合。仿真模型是在在第二三章节的状态空间方程导出的基础上建立的。
	这些模型将有助于实现后续实时控制器的设计以及对于整体系统的分析。
	
\end{itemize}
