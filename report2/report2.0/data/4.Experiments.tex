\clearpage
\section{实验阶段}

\subsection{实验参数设定}

表 \ref{tab:param} 列出本模型的相关参数及其数学标记。本报告中表示左右参数则分别以 $ \_L $ 和 $ \_R $表示。参数表主要分为分为整体系统参数,手推轮椅主体,机电驱动模块以及手推控制输入部分的参数四个部分。如下所示(见下页):

\begin{table}[H]
	\footnotesize
	\caption{系统主要参数及其数学标记}\label{tab:param}
	\begin{longtable}{l|l|l}
		\toprule
		\textbf{数学标记} & \textbf{系统参数} & \textbf{设计值}\\
		\midrule
		\endhead
		\multicolumn{3}{l}{\textbf{系统整体参数}} \\ % {占用行数} {文字居左中右} {内容}
		\midrule
		$ V_{\rm{CG}} $ & 质心速度 & \\
		$ w_{\rm{CG}} $ & 质心转速 & \\
		$ P_{\rm{CG}} $ & 系统动量变化率 & \\
		$ P_{\theta} $ & 系统角动量变化率 & \\
		\midrule
		\multicolumn{3}{l}{\textbf{手推轮椅主体}} \\
		\midrule
		SE:$\tau$ & 后轮推进扭矩($ \mathrm{N} \cdot \mathrm{m} $) & 12 \\
		$ J w $ & 后轮转动惯量($ \mathrm{kg} \cdot \mathrm{m}^2 $) & 0.005 \\
		$ M_t $ & 系统质量 & \\
		$ J t $ & 系统转动惯量 & \\
		$ R_g $ & 轮胎与地面摩擦系数 & \\
		$ C_w $ & 轮辐弹性系数 & \\
		$ R_w $ & 轮辐阻尼 & \\
		$ r $ & 后轮半径 & \\
		$ L_w $ & 轮椅宽度 & \\
		$ R_{wc} $ & 联轴器阻尼系数 & \\
		$ C_{wc} $ & 联轴器弹性系数 & \\
		\midrule
		\multicolumn{3}{l}{\textbf{机电驱动模块}} \\
		\midrule
		$ I_e $ & 电机电感 & \\
		$ R_e $ & 电机内阻 & \\
		$ J_r $ & 转子转动惯量 & \\
		$ R_b $ & 电机轴承阻尼 & \\
		$ M_d $ & 机电驱动模块质量 & \\
		$ J_d $ & 机电驱动模块转动惯量 & \\
		$ L_d $ & 模块宽度 & \\
		$ C_s $ & 电机输出转矩 & \\
		$ R_s $ & 电机输出轴阻尼 & \\
		$ k_1 $ & 电机扭矩系数 & \\ % 简化了k8
		$ k_2 $ & 齿轮系数比 & \\ % 简化了k7
		$ k_3 $ & 电动车轮半径 & \\ % 简化了k6
		Se:L & 输入控制电压\\
		\midrule
		\multicolumn{3}{l}{\textbf{手推控制输入部分}}\\
		\midrule
		$ J $ & 手动轮转动惯量 & \\
		$ k_4 $ & 手动轮半径 & \\ % 简化了k10
		\bottomrule
	\end{longtable}
\end{table}